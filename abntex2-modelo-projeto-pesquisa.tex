% ------------------------------------------------------------------------
% ------------------------------------------------------------------------
% abnTeX2: Modelo de Projeto de pesquisa em conformidade com 
% ABNT NBR 15287:2011 Informação e documentação - Projeto de pesquisa -
% Apresentação 
% ------------------------------------------------------------------------ 
% ------------------------------------------------------------------------

\documentclass[
	% -- opções da classe memoir --
	arial,
	12pt,				% tamanho da fonte
	openright,			% capítulos começam em pág ímpar (insere página vazia caso preciso)
	%twoside,			% para impressão em recto e verso. Oposto a oneside
	oneside,
	a4paper,			% tamanho do papel. 
	% -- opções da classe abntex2 --
	chapter=TITLE,		% títulos de capítulos convertidos em letras maiúsculas
	%section=TITLE,		% títulos de seções convertidos em letras maiúsculas
	%subsection=TITLE,	% títulos de subseções convertidos em letras maiúsculas
	%subsubsection=TITLE,% títulos de subsubseções convertidos em letras maiúsculas
	% -- opções do pacote babel --
	english,			% idioma adicional para hifenização
	french,				% idioma adicional para hifenização
	spanish,			% idioma adicional para hifenização
	brazil,				% o último idioma é o principal do documento
	]{abntex2}

% ---
% PACOTES
% ---

% ---
% Pacotes fundamentais 
% ---
\usepackage{lmodern}			% Usa a fonte Latin Modern
\usepackage[T1]{fontenc}		% Selecao de codigos de fonte.
\usepackage[utf8]{inputenc}		% Codificacao do documento (conversão automática dos acentos)
\usepackage{indentfirst}		% Indenta o primeiro parágrafo de cada seção.
\usepackage{color}				% Controle das cores
\usepackage{graphicx}			% Inclusão de gráficos
\usepackage{microtype} 			% para melhorias de justificação
% ---

% ---
% Pacotes adicionais, usados apenas no âmbito do Modelo Canônico do abnteX2
% ---
\usepackage{lipsum}				% para geração de dummy text
% ---

% ---
% Pacotes de citações
% ---
\usepackage[brazilian,hyperpageref]{backref}	 % Paginas com as citações na bibl
\usepackage[alf]{abntex2cite}	% Citações padrão ABNT

% --- 
% CONFIGURAÇÕES DE PACOTES
% --- 

% ---
%%% Exemplo de \newcommand no corpo do documento.
	%%% Embora possível, é recomendável que todas as definições do
	%%% usuário fiquem reunidas no preâmbulo ou ainda num package.
	%\newcommand{\X}{\checked}
	%\newcommand{\X}{\textbullet}
\newcommand{\X}{\textbullet}

% ---
% Configurações do pacote backref
% Usado sem a opção hyperpageref de backref
\renewcommand{\backrefpagesname}{Citado na(s) página(s):~}
% Texto padrão antes do número das páginas
\renewcommand{\backref}{}
% Define os textos da citação
\renewcommand*{\backrefalt}[4]{
	\ifcase #1 %
		Nenhuma citação no texto.%
	\or
		Citado na página #2.%
	\else
		Citado #1 vezes nas páginas #2.%
	\fi}%
% ---

% ---
% Informações de dados para CAPA e FOLHA DE ROSTO
% ---
\titulo{\uppercase{ A Experiência de Duas Instituições de Ensino Superior do Distrito Federal em Manter o Ensino Disruptivo em Meio a Pandemia do COVID-19: um Estudo de Caso }}
\autor{Danrley Willyan da Silva Pereira \and Eliel Cruz da Silva }
\local{Brasília}
\data{\the\year}
\instituicao{%
 Centro Universitário UDF
  \par
  Engenharia e Tecnologia
  \par
  Programa de Iniciação Científica}
\tipotrabalho{Projeto de Pesquisa}
% O preambulo deve conter o tipo do trabalho, o objetivo, 
% o nome da instituição e a área de concentração 
\preambulo{Projeto de pesquisa submetido ao edital de iniciação científica de 2020 do Centro Universitário do Distrito Federal (UDF).}
% ---

% ---
% Configurações de aparência do PDF final

% alterando o aspecto da cor azul
\definecolor{blue}{RGB}{41,5,195}

% informações do PDF
\makeatletter
\hypersetup{
     	%pagebackref=true,
		pdftitle={\@title}, 
		pdfauthor={\@author},
    	pdfsubject={\imprimirpreambulo},
	    pdfcreator={LaTeX with abnTeX2},
		pdfkeywords={covid19}{ead}{projeto de pesquisa}, 
		colorlinks=true,       		% false: boxed links; true: colored links
    	linkcolor=blue,          	% color of internal links
    	citecolor=blue,        		% color of links to bibliography
    	filecolor=magenta,      		% color of file links
		urlcolor=blue,
		bookmarksdepth=4
}
\makeatother
% --- 

% --- 
% Espaçamentos entre linhas e parágrafos 
% --- 

% O tamanho do parágrafo é dado por:
\setlength{\parindent}{1.3cm}

% Controle do espaçamento entre um parágrafo e outro:
\setlength{\parskip}{0.2cm}  % tente também \onelineskip

% ---
% compila o indice
% ---
\makeindex
% ---

% ----
% Início do documento
% ----
\begin{document}

% Seleciona o idioma do documento (conforme pacotes do babel)
%\selectlanguage{english}
\selectlanguage{brazil}

% Retira espaço extra obsoleto entre as frases.
\frenchspacing 

% ----------------------------------------------------------
% ELEMENTOS PRÉ-TEXTUAIS
% ----------------------------------------------------------
% \pretextual

% ---
% Capa
% ---
\imprimircapa
% ---

% ---
% Folha de rosto
% ---
\imprimirfolhaderosto
% ---

% ---
% NOTA DA ABNT NBR 15287:2011, p. 4:
%  ``Se exigido pela entidade, apresentar os dados curriculares do autor em
%     folha ou página distinta após a folha de rosto.''
% ---

% --- resumo em português ---
\begin{resumo}
Uma pandemia mundial está forçando as escolas e faculdades a fecharem suas portas. No entanto, manter a educação continua necessário, de maneira disruptiva, é claro. Como as faculdades podem manter o ensino disruptivo em meio a pandemia ? A única forma viável é através da educação a distância, principalmente usando os ambientes virtuais de aprendizagem. A educação a distância é um tema recorrente em estudos acadêmicos, existindo diferentes técnicas de aplicação. O COVID-19 é uma crise compartilhada entre todo o mundo e os seus impactos socioeconômicos precisam ser estudados de diferentes perspectivas. Analisar a eficiência do ensino durante a pandemia do COVID-19 com base na literatura existente é o principal foco desse estudo.
\vspace{\onelineskip}

\noindent
\textbf{Palavras-chave}: covid-19. ead. educação a distância. pandemia. projeto de pesquisa. 
\end{resumo}
% --- resumo em francês ---
\begin{resumo}[Abstract]
\begin{otherlanguage*}{english}
A worldwide pandemic is forcing faculties to close their doors. Yet the need to teach remains, on a disruptive way of course. How can faculties mantain the disruptive learning in a time of pandemic ? The only way viable is through distance education, mainly using virtual learning environments. The distance education is a recurrent theme at academic studies, having different usage techniques. The COVID-19 is a world shared crisis and its socioeconomics impacts have to be study using different perspectives. Analise the efficiency of that learn at COVID-19 pandemic accordingly to existing literature is the main focus of this proposal.
\vspace{\onelineskip}

\noindent
\textbf{Keywords}:  covid-19. ead. distance education. pandemic. research project. 
\end{otherlanguage*}
\end{resumo}

% ---
% inserir lista de ilustrações
% ---
\pdfbookmark[0]{\listfigurename}{lof}
%\listoffigures*
\cleardoublepage
% ---

% ---
% inserir lista de tabelas
% ---
\pdfbookmark[0]{\listtablename}{lot}
\listoftables*
\cleardoublepage
% ---

% ---
% inserir lista de abreviaturas e siglas
% ---
\begin{siglas}
  \item[COVID-19] Coronavirus Disease 2019
  \item[DF] Distrito Federal
  \item[EAD] Educação a Distância
  \item[IES] Instituição de Ensino Superior
  \item[SARS-CoV-2] Severe Acute Respiratory Syndrome Coronavirus 2
  \item[SARS] Severe Acute Respiratory Syndrome
  \item[TIC] Tecnologias de Informação e Comunicação
  \item[UDF] Centro Universitário do Distitro Federal
  \item[UNESCO] United Nations Educational, Scientific and Cultural Organization
\end{siglas}
% ---

% ---
% inserir lista de símbolos
% ---
%\begin{simbolos}
  %\item[$ \Gamma $] Letra grega Gama
  %\item[$ \Lambda $] Lambda
  %\item[$ \zeta $] Letra grega minúscula zeta
  %\item[$ \in $] Pertence
%\end{simbolos}
% ---

% ---
% inserir o sumario
% ---
\pdfbookmark[0]{\contentsname}{toc}
\tableofcontents*
\cleardoublepage
% ---


% ----------------------------------------------------------
% ELEMENTOS TEXTUAIS
% ----------------------------------------------------------
\textual

% ----------------------------------------------------------
% Introdução
% ----------------------------------------------------------
%\chapter*[Introdução]{Introdução} \label{introduction}
\chapter{Introdução}\label{introduction}
%\addcontentsline{toc}{chapter}{Introdução}

O presente projeto de pesquisa tem como tema o uso da Educação a Distância (EAD)\index{EAD} em tempos de pandemia \index{pandemia} do \textit{coronavirus disease 2019} (COVID-19) \index{COVID-19}. Os estudantes do ensino superior \index{ensino!superior} presencial 
estão enfrentado um momento delicado, o isolamento social \index{isolamento social}, ou seja, eles precisam modificar hábitos para se adaptarem à EAD \cite{zhou2020school}; e para passar por esse momento eles precisam de apoio da Instituição de 
Ensino Superior (IES) na qual estão matriculados \cite{xie2020autonomous}. Como as IES's do Distrito Federal (DF) estão auxiliando os estudantes a continuarem construindo seu conhecimento nesse 
momento de disrupção da educação e como as próprias IES's estão adaptando sua forma de ensino, seus materiais e seus departamentos de atendimento aos estudantes para melhor atendê-los ?

Visando a viabilidade técnica e econômica do estudo, separou-se uma amostra de duas IES entre as 64 IES's do DF, dados adquiridos pelo censo da educação superior \index{censo da educação superior} realizado pelo \index{INEP}\citeonline{censoInep}. As duas 
instituições de ensino, ``Instituição N'' e ``Instituição O'', já foram contactadas através de membros do corpo docente para participarem da pesquisa e para auxiliar na divulgação dos questionários para o corpo discente. 
Ambas instituições se encontram em locais de fácil acesso e não oferecem risco técnico e financeiro a realização de visitas e reuniões com os participantes da pesquisa se necessário. \index{corpo!docente} \index{corpo!discente}

A hipótese desse estudo é que as IES's têm tecnologias/softwares como meio de disponibilizar vídeos, materiais, fazer video chamadas entre os professores e os alunos, disponibilizar links e recursos educacionais\index{recurso educacional},
ou seja, elas têm um ambiente virtual de aprendizagem\index{ambiente virtual de aprendizagem}. Mas, infelizmente, possibilitar a EAD e o aprendizado autônomo dos estudantes exige mais do que ter um ambiente virtual de aprendizagem, é necessário que as IES's 
desenvolvam métodos de aconselhar academicamente os estudantes visando a efetivação e, posteriormente, a qualidade do estudo domiciliar \cite{xie2020autonomous}. Essa habilidade de aconselhar academicamente
 será um dos pontos chave de sucesso ou falha das IES's em manter o ensino disruptivo \index{ensino!disruptivo} em meio a pandemia \index{pandemia} do COVID-19. 

O \textit{objetivo geral} da proposta aqui apresentada é identificar através de métodos quantitativos e qualitativos \index{abordagem!quantitativa} \index{abordagem!qualitativa} se as IES's estudadas conseguiram manter o ensino de maneira disruptiva durante o isolamento social\index{isolamento social} imposto focando o combate à pandemia \index{pandemia} do COVID-19. Para isso, na visão dos pesquisadores e baseado na literatura existente, criou-se alguns objetivos específicos, tais objetivos são enumerados a seguir:

\begin{enumerate}
  \item \label{obj-esp1} \textit{Específico 01}: Analisar se os ambientes virtuais de aprendizagem\index{ambiente virtual de aprendizagem} das IES's estudadas são de fácil uso e se eles são capazes de ser um instrumento de ampliação de oportunidades educacionais \cite{elielCruz}.
  \item \label{obj-esp2} \textit{Específico 02}: Identificar se houveram ações de aconselhamento acadêmico com foco em cultivar a habilidade de aprender em casa e melhorar a qualidade dos estudos em casa.
  \item \label{obj-esp3} \textit{Específico 03}: Comparar de maneira quantitativa o aprendizado dos estudantes no ano de 2019 e 2020 (análise estatística).
  \item \label{obj-esp4} \textit{Específico 04}: Analisar se os métodos normais de ensino foram simplesmente copiados ou se houve um trabalho de adaptação dos materiais e conteúdos para o ambiente virtual de aprendizagem;
  \item \label{obj-esp5} \textit{Específico 05}: Alunos bolsistas ou usufrutuários de programas governamentais conseguiram usufruir dos recursos educacionais\index{recurso educacional} e efetivar a aprendizagem através do ambiente virtual de aprendizagem ?
  \item \label{obj-esp6} \textit{Específico 06}: Enquadrar o tipo de abordagem de cada instituição para o uso da EAD: \textit{broadcast}, virtualização da sala de aula presencial (ensino síncrono remoto) ou \textit{estar junto social}.

\end{enumerate}

COVID-19\index{COVID-19} é um termo que está sendo amplamente pesquisado no \index{Brasil}Brasil e no mundo \cite{googleTrends} durante os primeiros meses de 2020, no entanto, vale ressaltar, essa pesquisa está sendo geralmente realizada atrelada à saúde pessoal e coletiva. Porém, no atual momento de isolamento social\index{isolamento social} causado pela pandemia\index{pandemia} do COVID-19, é necessário fazer pesquisa científica atrelada as consequências sociais e políticas do isolamento, tema qual o presente trabalho busca chegar a um entendimento. 

O uso de EAD\index{EAD} ou, até mesmo, ensino remoto síncrono em IES's com foco em mitigar as consequências sociais (educacionais) em meio a pandemia\index{pandemia} é um tema que tem que ser estudado para possibilitar referências para estudos futuros, principalmente os estudos que irão focar nas consequências e impactos para o ensino superior \index{ensino!superior} brasileiro; além de, é claro, saber se é viável o uso desses métodos de ensino-aprendizado \index{ensino!-aprendizagem} em momentos de crise.Vale ressaltar que China\index{China}, Estados Unidos da América\index{Estados Unidos da América}, \index{Filipinas}Filipinas, \index{Indonésia}Indonésia, dentre outros, além de organizações multilaterais como a United Nations Educational, Scientific and Cultural Organization (UNESCO)\index{UNESCO}, já começaram a publicar artigos sobre o tema educação e COVID-19 no mês de março e abril de 2020:

\begin{itemize}

  \item Autonomous learning of elementary students at home during the COVID-19 epidemic: A case study of the second elementary school in Daxie, Ningbo, Zhejiang Province, China \cite{xie2020autonomous};
  \item Replacing the Classic Learning Form at Universities as an Immediate Response to the COVID-19 Virus Infection in Georgia \cite{georgia};
  \item `School’s Out, But Class’ On', The Largest Online Education in the World Today: Taking China’s Practical Exploration During The COVID-19 Epidemic Prevention and Control As an Example \cite{zhou2020school};\index{``School’s Out, But Class’ On''}
  \item Sentiment Analysis on Synchronous Online Delivery of Instruction due to Extreme Community Quarantine in the Philippines caused by COVID-19 Pandemic \cite{pastor2020sentiment};
  \item Using Technology to Maintain the Education of Residents During the COVID-19 Pandemic \cite{chick2020using}
  \item The Impact of Covid-19\index{COVID-19} to Indonesian Education and Its Relation to the Philosophy of “Merdeka Belajar” \cite{abidah2020impact};
  \item Handbook on Facilitating Flexible Learning During Educational Disruption: The Chinese Experience in Maintaining Undisrupted Learning in COVID-19 Outbreak \cite{huang2020handbook};
  \item Education Emergencies \cite{unesco};\index{UNESCO}
  \item Public-Private Virtual-School Partnerships and Federal Flexibility for Schools during COVID-19 \cite{butcher2020public}.

\end{itemize}

Tantos artigos científicos sendo publicados em tão pouco tempo é um sinal claro da relevância de um estudo tal como o proposto aqui. 

% ----------------------------------------------------------
% Capitulo de Referencial Teórico
% ----------------------------------------------------------
\chapter{Revisão de Literatura}

Como já mencionado, o presente projeto de pesquisa busca entender o uso e a qualidade do uso da EAD\index{EAD} e de métodos de ensino remoto síncrono no momento de pandemia\index{pandemia} do COVID-19, com isso em mente fez-se uma expressão geral de busca de trabalhos acadêmicos em sítios como \url{http://www.scielo.com}, \url{http://scholar.google.com}, \url{https://www.researchgate.net}, \url{http://www.scielo.com} e \url{http://www.periodicocapes.com}; a expressão ``Educação a distância AND COVID-19'' não se fez útil na tentativa de encontrar trabalhos acadêmicos em português. Sem êxito, buscou-se a mesma expressão em inglês, obtendo melhor resultado. Como era de se esperar, os resultados são trabalhos acadêmicos recentes, todos datados do ano de 2020.  \index{COVID-19}

Além dessa expressão de busca, fez-se necessário expressões de buscas como: ``\textit{Viable alternative} AND \textit{distance education}'' e ``COVID-19 OR \textit{Distance education}''. Essas expressões fizeram-se necessárias para que os pesquisadores possam fazer um levantamento bibliográfico sobre cada tema separado, buscando-se propriedade sobre o tema de interesse do trabalho, visando sempre uma análise mais acurada para se obter maior qualidade no resultado da pesquisa.

\section{COVID-19}

COVID-19 é uma infecção respiratória causada pelo vírus\index{vírus} \textit{Severe Acute Respiratory Syndrome Coronavirus 2} (SARS-CoV-2)\index{SARS-CoV-2} \cite{desai2020stopping}, vírus que foi descoberto recentemente depois de contaminar em massa pessoas no estado de Wuhan-China\index{China}, em dezembro de 2019. SARS-CoV-2 pertence a uma família de vírus (corona-virus)\index{corona} que causam moléstias que vão desde o resfriado comum até as mais severas infecções em humanos.

COVID-19 causa uma variedade de sintomas em pessoas que foram infectadas, e nem todas as pessoas infectadas terão os mesmo sintomas. Febre, tosse seca, dificuldade de respirar e fadiga são alguns dos sintomas mais comuns; entretanto, algumas pessoas apresentaram dores de cabeça, dores abdominais, diarreia e garganta inflamada \cite{desai2020stopping}. Muitas pessoas desenvolvem sintomas brandos ou nenhum sintoma; as mortes são associadas principalmente à pessoas idosas e/ou com doenças crônicas\index{doenças crônicas} \cite{watkins2020preventing}. 

Modelos epidemiológicos\index{epidemiologia} básicos do avanço do vírus SARS-CoV-2 sugerem que, devido a sua alta contagiosidade e a falta de imunidade da população, 40 a 70 porcento da população poderá ser infectada a não ser que medidas duras sejam tomadas \cite{salathe2020covid}. Dois vírus\index{vírus} tiveram suas deflagrações encurtadas por uma política de \textit{catch-and-isolate} (pegar-e-isolar), o \textit{Severe Acute Respiratory Syndrome} (SARS)\index{SARS} em 2003 e o H1N1 (Influenza A)\index{H1N1} em 2009 \cite{watkins2020preventing}, ou seja, isolar as pessoas que foram contaminadas por esses vírus\index{vírus} foi eficaz para ambos os casos. Infelizmente, o COVID-19, mesmo adotando-se essa política mundo afora, continuou se espalhando, exigindo medidas mais drásticas dos governos, isolamento e distanciamento social\index{isolamento social}, fechando-se fronteiras e colocando países inteiros em quarentena. 

Baseado em dados obtidos pela \citeonline{unesco}, mais de 160 países implementaram medidas nacionais contra o COVID-19\index{COVID-19}, afetando mais da metade da população mundial de estudantes. As últimas estatísticas da UNESCO\index{UNESCO} (20/04/2020) mencionava que mais de 1.6 bilhões de estudantes foram afetados pelo COVID-19, isso é quase 90 porcento do total de estudantes matriculados em instituições de ensino. A UNESCO está dando suporte direto aos países, incluindo soluções de ensino à distância para assegurar a continuidade do ensino para todos.

\section{EAD}

O desenvolvimento científico acelerado, associado ao crescente aumento da capacidade tecnológica, ampliou as possibilidades de atuação na área da educação. Dentre as possibilidades, temos a EAD como uma das que mais cresceram \cite{jorgeEad}. Para o presente estudo considerar-se-á a concepção de \citeonline{landim} para EAD, ou seja, a EAD pressupõe a combinação de tecnologias que possibilitem o estudo em qualquer lugar, por meio de métodos de orientação e tutoria à distância.

O trabalho de \citeonline{jorgeEad}, deixa claro que o processo de ensino-aprendizagem\index{ensino!-aprendizagem} através da EAD só é concretizado quando o aluno tem habilidades específicas e o domínio das ferramentas utilizadas para transmissão do conhecimento. 

\begin{citacao}
Um dos pontos considerado positivo da EAD\index{EAD} é a possibilidade do acesso à informação a pessoas diversas em locais diversos geograficamente. No entanto, o processo de ensino-aprendizagem requer habilidades diferenciadas na apresentação, planejamento, desenvolvimento e avaliação da aprendizagem, bem como o domínio das ferramentas de transmissão a serem utilizadas. \cite{jorgeEad}
\end{citacao}  

Até pouco tempo atrás as Tecnologias de Informação e Comunicação (TIC\index{TIC}) não eram amplamente usadas no ensino superior\index{ensino!superior}, sendo utilizadas para propostas de ensino não presencial. Hoje em dia, essas tecnologias foram incorporadas na dinâmica do ensino universitário, gerando o que conhecemos como ambientes virtuais de aprendizagem\index{ambiente virtual de aprendizagem}. Esses ambientes são sistemas computacionais disponíveis na internet, que dão suporte as atividades utilizando-se as TIC, de acordo com o estudo de \citeonline{almeida2003educaccao} eles permitem ``integrar múltiplas mídias, linguagens e recursos, apresentar informações de maneira organizada, desenvolver interações entre pessoas e objetos de conhecimento, elaborar e socializar produções tendo em vista atingir determinados objetivos''.


\section{Uso de EAD em meio a pandemia: uma alternativa}

A principal referência que temos ao uso da EAD nesse momento de isolamento social\index{isolamento social} é a China\index{China}, que já vem encarando o COVID-19\index{COVID-19} desde de dezembro de 2019. O governo chinês lançou uma campanha nomeada ``School’s Out, But Class’ On'' durante a pandemia\index{pandemia}; \citeonline{zhou2020school} analizou a fundamentação e os possíveis impactos da campanha, focando nas recomendações do \index{Ministério da Educação da China}Ministério da Educação da China e as estratégias necessárias para que a campanha seja bem sucedida. Outro documento de caráter científico que podemos usar como referência é um estudo de caso de uma escola em Daxie-Ningbo na \index{Província de Zhejiang}Província de Zhejiang-China \cite{xie2020autonomous}, um estudo de caso da campanha ``School’s Out, But Class’ On''\index{``School’s Out, But Class’ On''}.

A partir dessas referências e devido as circuntâncias, é claro que a EAD\index{EAD} é uma alternativa viável em meio a pandemia\index{pandemia}, porém se faz necessário uma análise acurada para verificar se a construção do conhecimento foi efetuada. Essa análise tem que ser feita através de diferentes perspectivas:

 \begin{itemize}

  \item Tecnologia utilizada \cite{benakouche2000educaccao, almeida2003educaccao, georgia};
  \item Adaptação dos conteúdos para a EAD \cite{xie2020autonomous, zhou2020school, benakouche2000educaccao};
  \item Capacitação docente para uso da EAD \cite{benakouche2000educaccao};
  \item Habilidade dos discentes (tecnológica e socioemocional) \cite{oranburg2020distance, arieira2009avaliaccao};
  \item Estratégias de orientação \cite{xie2020autonomous, reich2020remote, zhou2020school}.

\end{itemize}

Tendo em vista todos os estudos citados, a constatação da continuidade do ensino e a verificação da qualidade do uso da EAD podem ser feitas através de um estudo de caso em instituições reais.


% ----------------------------------------------------------
% Capitulo de Metodologia
% ----------------------------------------------------------
\chapter{Metodologia}

Haja vista a necessidade de \index{isolamento social}isolamento social para diminuir o avanço do COVID-19\index{COVID-19}, a EAD\index{EAD} está sendo a escolha óbvia para manter o ensino disruptivo nas IES's do \index{Brasil}Brasil e do Mundo. No entanto, é necessário fazer uma análise baseada na literatura existente sobre o assunto para descobrir se as IES's conseguiram manter o \index{ensino!disruptivo}ensino disruptivo meio a pandemia\index{pandemia}. Para realizar essa análise o presente estudo utilizará uma abordagem quanti-qualitativa \index{abordagem!quantitativa} \index{abordagem!qualitativa} \cite{elielCruz} para se obter dados coletados por meio de um questionário estruturado com questões objetivas (fechadas) e subjetivas (abertas) que serão desenvolvidas com base na revisão de literatura desse trabalho e aplicadas nas insituições ``O'' e ``N'', duas IES's do DF que formam a amostra(???) da pesquisa.

A decisão de utilizar as abordagens qualitativa e quantitativa é baseada no fato de que uma única abordagem metodológica não consegue alcançar os distintos objetivos da pesquisa em ciências humanas \cite{santos}, essa decisão possibilita a complementaridade essencial para esse tipo de pesquisa. Através do uso de questionários\index{questionário} é possível a coleta de dados que podem ser analisados de forma qualitativa e quantitativa de acordo com \citeonline{elielCruz}, além de que a aplicação de questionários via formulários do Google\index{Google Forms} é técnicamente a única alternativa para se obter as respostas frente à realidade de distanciamento e isolamento social\index{isolamento social} que se enfrenta no DF. Os questionários serão aplicados ao corpo docente e discente das duas instituições. \index{corpo!docente} \index{corpo!discente}

Os dados obtidos serão analisados sob a ótica da literatura visando responder os  objetivos de pesquisa enumerados na \nameref{introduction}. Esse trabalho fará uso de análises estatísticas para comparar de forma quantitativa (dados númericos) o aprendizado dos estudantes no ano de 2019 e 2020, visando alcançar o \textit{objetivo específico 0\ref{obj-esp3}}. O \textit{objetivo específico 0\ref{obj-esp1}} será abordado tendo como referência os estudos de \citeonline{almeida2003educaccao}, \citeonline{georgia} e \citeonline{elielCruz}, dessa forma obtera-se-á um resultado mais acurado, já que a análise não será feita com base em somente um estudo. 

Os \textit{objetivos específicos 0\ref{obj-esp2} e 0\ref{obj-esp4}} foram pensados com base nos trabalhos de \citeonline{zhou2020school}, \citeonline{reich2020remote}, \citeonline{huang2020handbook} e \citeonline{xie2020autonomous} e serão análisados com base nos mesmos. O \textit{objetivo específico 0\ref{obj-esp5}} se faz necessário devido as consequências socioeconômicas da crise deflagrada pela pandemia\index{pandemia} do COVID-19, além de alguns estudos sobre EAD que deixam claro os principais problemas relacionados ao seu uso, como o acesso à internet e à ferramentas necessárias para usufruir dessa modalidade de ensino \cite{benakouche2000educaccao, arieira2009avaliaccao}.

Conforme \citeonline{pradoValente} existem três abordagens de EAD por meio das TIC\index{TIC}: \textit{broadcast}, virtualização da sala de aula presencial (ensino síncrono remoto) ou \textit{estar junto social}. Na abordagem \textit{broadcast} a tecnologia computacional é usada para entregar informações aos estudantes como ocorre no rádio e na televisão. A sala de aula presencial é o que denominamos nesse estudo de ensino remoto síncrono, o que exige o paradigma do espaço-tempo, ou seja, uma comuicação bidirecional entre professor e aluno que se encontram no mesmo horário e no mesmo espaço (ferramenta de videoconferência). O \textit{estar junto social} explora a interatividade das TIC\index{TIC} por meio da comunicação multidimensional \cite{almeida2003educaccao}, o que permiti criar condições de aprendizagem e colaboração. O \textit{objetivo específico 0\ref{obj-esp6}} será analisado com base na perspectiva desses autores.

A partir dessas análises discorrer-se-á sobre a efetiva qualidade do uso da EAD durante a pandemia do COVID-19\index{COVID-19}, ficando claro se as IES's estudadas foram capazes de manter o ensino disruptivo durante o isolamento social\index{isolamento social} e se o uso da EAD\index{EAD} é uma alternativa viável para as IES's em momentos de crise.

% ----------------------------------------------------------
% Capitulo de Cronograma
% ----------------------------------------------------------
\chapter{Cronograma  e Recursos}

Esse projeto de pesquisa está sendo proposto por um aluno e um professor do curso de Análise e Desenvolvimento de Sistemas do Centro Universitário do Distitro Federal (UDF). Danrley Pereira será o aluno pesquisador do projeto de Iniciação Científica da UDF com a orientação do professor Eliel Cruz, ambos trabalharão juntos para a elaborar dois questionários, um para o corpo docente e outro para o corpo discente, visando a obtenção de dados acurados e fidedignos através de uma ferramenta gratuita chamada da \index{Google Forms}\textit{Google Forms}.

Poucos são os recursos materiais e financeiros necessários à realização desse trabalho de pesquisa, pois os questionários serão aplicados através da rede mundial de computadores (internet), sendo necessário uma ou duas visitas às instituições ``O'' e ``N'' para alinhar o plano de obtenção dos dados e para analisar as questões éticas e morais atreladas à natureza dessa proposta. A própria ferramenta utilizada para aplicação dos questionários já faz uma análise estatística, no entanto, para uma análise mais acurada utilizar-se-á bibliotecas gratuitas de análise de dados da linguagem de programação \index{Python}Python. 

 Abaixo estão enumeradas as ações necessárias para realização do trabalho de pesquisa, as ações são apresentadas na ordem que começarão a ser realizadas.

\begin{enumerate}
	\item \label{rev-lit} Revisão de literatura;
	\item \label{ela-quest} Elaboração dos questionários com base na literatura;
	\item \label{vis-inst}  Visita as instituições da amostra;
	\item \label{aplic-quest} Aplicação dos questionários para o corpo docente e discente;
	\item \label{anI} Análise quantivativa (estatística) dos dados obtidos;
	\item \label{anII} Análise qualitativa dos dados obtidos;
	\item \label{rp} Relatório parcial junto à primeira versão do trabalho;
	\item \label{pf} Relatório final e versão corrigida do trabalho.
\end{enumerate}

A revisão de literatura deverá ser feita pelo aluno pesquisador, buscando usar alguns elementos de revisão sistemática de literatura, através de um protocolo de revisão. Em paralelo à revisão de literatura, fazer-se-á os questionários, haja vista que os questionários são baseados na literatura e serão de mais fácil elaboração durante a revisão. As visitas as duas IES's por ambos os pesquisadores (aluno e orientador), ``Instituição N'' e ``Instituição O'', busca a transparência entre os envolvidos nesse trabalho de pesquisa, não obstante, facilitar a ação número 04. Como já explicitado, a análise quantitativa deverá ser realizada usando-se a linguagem de programação Python\index{Python}. A análise qualitativa deverá ser feita com maior atenção por parte dos pesquisadores, pois essa é a análise necessária para se alcançar a maioria dos objetivos específicos desse trabalho, não sendo realizada corretamente, põem-se em risco todo o estudo. A versão parcial do trabalho deverá ser entregue para devida análise e correção do professor orientador, junto a membros do corpo docente interessados; já a versão final virá acompanhada de uma apresentação oral e visual para o evento anual de iniciação científica da UDF.

	%%% Exemplo de \newcommand no corpo do documento.
	%%% Embora possível, é recomendável que todas as definições do
	%%% usuário fiquem reunidas no preâmbulo ou ainda num package.
	%\newcommand{\X}{\checked}
	%\newcommand{\X}{\textbullet}
	
	A Tab.~\ref{tab:bullets} mostra o cronograma que deve ser seguido para realização desse trabalho. O símbolo
	`\X' foi usado para mostrar os respectivos meses estimados para cada ação.
	
	\begin{table}[htbp]
	  \centering
	  \begin{tabular}{|l||c|c|c|c|c|c|c|c|}
	    \hline
	    \multicolumn{9}{|c|}{\textbf{2020-2021}} \\
	    \hline
	    \hline
	    Fase  & Agosto & Setembro & Outubro & Novembro & Dezembro & Janeiro & Fevereiro &  Março \\
	    \hline
	    \ref{rev-lit}     & \X & \X & \X &     &      &      &     &     \\
	 \hline
	    \ref{ela-quest}     &     & \X & \X &     &      &      &     &      \\
	\hline
	    \ref{vis-inst}     &     &     & \X & \X  &      &     &     &      \\
	\hline
	    \ref{aplic-quest}     &     &     &     & \X &  \X &      &     &      \\
	\hline
	    \ref{anI}     &     &     &      &      &     & \X &      &       \\
	\hline
	    \ref{anII}     &     &     &      &      &      & \X & \X &       \\
	\hline
	    \ref{rp}     &     &     &      &      &      &      & \X &       \\
	\hline
	    \ref{pf}     &     &     &      &      &      &      &     & \X   \\
	   % 9     &     &     &      &      &      &      &     & \X \\
	    \hline
	    \hline
	  \end{tabular}
	  \caption{Cronograma de Ação}%\eng{bullets}}
	  \label{tab:bullets}
	\end{table}

% ----------------------------------------------------------
% Capitulo com exemplos de comandos inseridos de arquivo externo 
% ----------------------------------------------------------

%\include{abntex2-modelo-include-comandos}

% ---
% Finaliza a parte no bookmark do PDF
% para que se inicie o bookmark na raiz
% e adiciona espaço de parte no Sumário
% ---
%\phantompart

% ---
% Conclusão
% ---
%\chapter*[Considerações finais]{Considerações finais}
\chapter{Considerações finais}
%\addcontentsline{toc}{chapter}{Considerações finais}

Essa situação que o mundo está vivendo é um momento impar na história recente, o impacto será e está sendo sentido e continuará impactando a economia e a sociedade por alguns anos. A relevância de se realizar pesquisa científica 
relacionadas a temas tão situacionais, mas que ficarão marcados na história da humanidade, é de extrema importância para que se entenda, no futuro próximo, as situações e dificuldades que diferentes setores da economia e sociedade passaram
durante a pandemia\index{pandemia} do COVID-19. \index{EAD}\index{COVID-19}

Com isso em mente, estudar o impacto do COVID-19 na educação, um importante setor social, é mais do que necessário. As dificuldades são inúmeras, mas não podemos desanimar diante desse cenário; a contribuição social da pesquisa proposta é evidente e se faz necessária para que a humanidade gere conhecimentos relacionados à pandemia\index{pandemia}, tendo sempre em vista o auxílio à sociedade e as gerações futuras que, com toda a certeza, enfrentarão crises parecidas. 

O objetivo desse projeto de pesquisa foi propor um trabalho de pesquisa necessário na área de educação, mais especificamente o uso da EAD\index{EAD} e do ensino remoto síncrono como fator de mitigação das consequências do isolamento social na vida do estudante do ensino superior\index{ensino!superior}. A partir da análise literaria, desenvolveu-se a hipótese de que a EAD é uma maneira de manter o ensino disruptivo de qualidade em meio a pandemia do COVID-19.



% ----------------------------------------------------------
% ELEMENTOS PÓS-TEXTUAIS
% ----------------------------------------------------------
\postextual

% ----------------------------------------------------------
% Referências bibliográficas
% ----------------------------------------------------------
\bibliography{abntex2-modelo-references}

% ----------------------------------------------------------
% Glossário
% ----------------------------------------------------------
%
% Consulte o manual da classe abntex2 para orientações sobre o glossário.
%
%\glossary

% ----------------------------------------------------------
% Apêndices
% ----------------------------------------------------------

% ---
% Inicia os apêndices
% ---
%\begin{apendicesenv}

% Imprime uma página indicando o início dos apêndices
%\partapendices

% ----------------------------------------------------------
%\chapter{Quisque libero justo}
% ----------------------------------------------------------

%\lipsum[50]

% ----------------------------------------------------------
%\chapter{Nullam elementum urna vel imperdiet sodales elit ipsum pharetra ligula
%ac pretium ante justo a nulla curabitur tristique arcu eu metus}
% ----------------------------------------------------------
%\lipsum[55-57]

%\end{apendicesenv}
% ---


% ----------------------------------------------------------
% Anexos
% ----------------------------------------------------------

% ---
% Inicia os anexos
% ---
%\begin{anexosenv}

% Imprime uma página indicando o início dos anexos
%\partanexos

% ---
%\chapter{Morbi ultrices rutrum lorem.}
% ---
%\lipsum[30]

% ---
%\chapter{Cras non urna sed feugiat cum sociis natoque penatibus et magnis dis
%parturient montes nascetur ridiculus mus}
% ---

%\lipsum[31]

% ---
%\chapter{Fusce facilisis lacinia dui}
% ---

%\lipsum[32]

%\end{anexosenv}

%---------------------------------------------------------------------
% INDICE REMISSIVO
%---------------------------------------------------------------------

\phantompart

\printindex


\end{document}
